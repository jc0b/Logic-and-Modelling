\documentclass{article}
\usepackage[utf8]{inputenc}
\usepackage{amsmath}
\usepackage{proof, environ, array}

\newcommand{\rulename}[1]{#1}

\title{Logic and Modelling Cheatsheet}
\author{Jacob Burley}
\date{Block 5 2018}

\begin{document}

\maketitle

\section*{Propositional Logic}

\subsection*{Introduction of $\wedge$}
    \begin{align*}
      \infer[\show\rulename{\wedge_i}]
      {\alpha \wedge \beta}
      {\alpha && \beta}
    \end{align*}
    (If you have derived $\alpha$ and $\beta$, then you can conclude $\alpha \wedge \beta$.)
\subsection*{Elimination of $\wedge$}
    \begin{align*}
      \infer[\rulename{\wedge_{e_1}}]
      {\alpha}
      {\alpha \wedge \beta}
      &&
      \infer[\rulename{\wedge_{e_2}}]
      {\beta}
      {\alpha \wedge \beta}
    \end{align*}

\subsection*{Introduction of $\vee$}
    \begin{align*}
      \infer[\rulename{\vee_{i_1}}]
      {\alpha \vee \beta}
      {\alpha}&&
      \infer[\rulename{\vee_{i_2}}]
      {\alpha \vee \beta}
      {\beta}
    \end{align*}
    
\subsection*{Elimination and introduction of $\neg\neg$}
    \begin{align*}
      \infer[\rulename{\neg\neg_e}]
      {\alpha}
      {\neg\neg\alpha}
      &&
      \infer[\rulename{\neg\neg_i}]
      {\neg\neg\alpha}
      {\alpha}
    \end{align*}
    
\subsection*{Elimination with $\to$}
    This rule is called ``Modus Ponens'' (MP):
    \begin{align*}
      \infer[\rulename{\to_e} (or \rulename{MP})]
      {\beta}
      {\alpha && \alpha \to \beta}
    \end{align*}
    (If we have derived $\alpha$ and we know that $\alpha\to\beta$, then you can conclude $\beta$.)
    \\\\This rule is called ``Modus Tollens'' (MT):
    \begin{align*}
      \infer[\rulename{MT}]
      {\neg\alpha}
      {\alpha \to \beta && \neg\beta}
    \end{align*}
    (If we have derived $\alpha \to \beta$ and we know that $\beta$ is false ($\neg\beta$), then you can conclude $\alpha$ is false ($\neg\alpha$).)
    
\subsection*{Introduction with $\to$}
    \begin{align*}
      \infer[\rulename{\to_i}]
      {\alpha \to \beta}
      {
        \framebox{\parbox{.8cm}{\centerline{$\alpha$}\centerline{$\vdots$}\centerline{$\beta$}}}
      }
    \end{align*}
    (Basically, if we make an assumption $\alpha$, and we can use the above rules to conclude $\beta$, then we can say that $\alpha\to\beta$.)
    
\subsection*{Copy rule}
    \begin{align*}
      \infer[\rulename{copy}]
      {\alpha}
      {\alpha}
    \end{align*}
    (Basically we can copy an $\alpha$ to where we want it to be if we would like to.)

\subsection*{Elimination and introduction with $\neg$}
    \begin{align*}
      \infer[\rulename{\neg_e}]
      {\bot}
      {\alpha && \neg \alpha}
    \end{align*}
    (Given a premise $\alpha$ and an instance of $\neg\alpha$, we can say that this is the same as $\bot$)
    
    \begin{align*}
      \infer[\rulename{\neg_i}]
      {\neg \alpha}
      {
        \framebox{\parbox{.8cm}{\centerline{$\alpha$}\centerline{$\vdots$}\centerline{$\bot$}}}
      }
    \end{align*}
    (If from a instance $\alpha$ we can derive $\bot$, we can say that this is the same as $\neg\alpha$)
    
\subsection*{Bottom (false) elimination}
    \begin{align*}
      \infer[\rulename{\bot_e}]
      {\alpha}
      {\bot}
    \end{align*}
    
\subsection*{Disjunction Elimination}
    \begin{align*}
      \infer[\rulename{\vee_e}]
      {\gamma}
      {
        \alpha \vee \beta && 
        \framebox{\parbox{.8cm}{\centerline{$\alpha$}\centerline{$\vdots$}\centerline{$\gamma$}}} 
        &&
        \framebox{\parbox{.8cm}{\centerline{$\beta$}\centerline{$\vdots$}\centerline{$\gamma$}}} 
      }
    \end{align*}

\subsection*{Proof by Contradiction / Reductio ad Absurdum}
    \begin{align*}
        \infer[\rulename{PBC} \quad (or \rulename{RAA})]
      {\alpha}
      {\framebox{\parbox{.8cm}{\centerline{$\neg\alpha$}\centerline{$\vdots$}\centerline{$\bot$}}}}
    \end{align*}

\subsection*{Law of Excluded Middle}
    \vspace{3ex}
    \begin{align*}
       \infer[\rulename{LEM}]
      {\alpha \vee \neg \alpha}
      {
      } 
    \end{align*}
    (This rule does not have premises)
\end{document}
